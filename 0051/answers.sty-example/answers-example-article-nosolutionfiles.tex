% !TEX program = LuaLaTeX
%%%%%%%%%%%%%%%%%%%%%%%%%%%%%%%%%%%%%%%%%%%%%%%%%%%%%%%%%%%%%%%%%%%%%%%%%%%%%%
\def\nosolutionfiles{} % 定義すると \usepackage[nosolutionfiles]{answers} になる.
\def\TITLE{answers.styの使用例}
\def\AUTHOR{黒木 玄}
\def\DATE{2024年12月31日}
%%%%%%%%%%%%%%%%%%%%%%%%%%%%%%%%%%%%%%%%%%%%%%%%%%%%%%%%%%%%%%%%%%%%%%%%%%%%%%
\documentclass[%
  lualatex,
  ja=standard,
  textwidth-limit=60,
  a4paper,
  %b5paper,
  12pt,
  oneside,
  everyparhook=compat]{bxjsarticle}

%%% amsthm
\usepackage{amsthm}
\theoremstyle{definition}
\newtheorem{problem}{問題}[section]

%%% answers
\ifdefined\nosolutionfiles
  \usepackage[nosolutionfiles]{answers}
\else
  \usepackage{answers}
\fi
\newcommand\answersfile{articleanswers}
\Newassociation{answer}{showanswer}{\answersfile}
\renewcommand{\showanswerlabel}[1]{\hspace{5.4pt}\textbf{問題#1解答例.}}

\newcommand\showanswersection{%
  \section{問題の解答例}
  \addcontentsline{toc}{section}{問題の解答例}
  \newcounter{answersection}
  \setcounter{answersection}{0}
}

\newcommand\showanswersubsection{%
  \stepcounter{answersection}
  \subsection{第\theanswersection 節の問題の解答例}
}

%%% maketitle
\title{\Huge{\textbf{\TITLE}}}
\author{\LARGE{\textbf{\AUTHOR}}}
\date{\Large{\textbf{\DATE}}}

%%%%%%%%%%%%%%%%%%%%%%%%%%%%%%%%%%%%%%%%%%%%%%%%%%%%%%%%%%%%%%%%%%%%%%%%%%%%%%
\begin{document}

%%% タイトルと著者など
\maketitle

%%% 目次
\setcounter{tocdepth}{3}
\tableofcontents

%%% 本文
\Opensolutionfile{\answersfile}
\begin{Filesave}{\answersfile}\showanswersection\end{Filesave}

\section{満月に注意せよ}
\begin{Filesave}{\answersfile}\showanswersubsection\end{Filesave}

\subsection{深夜でも明るい}

\subsubsection{散歩するとどうなるか}

\begin{problem}
満月の深夜に近所の神社に散歩に行くとどうなりますか?
\begin{answer}
満月の深夜に近所の神社に散歩に行くと, 警官に見付かって職務質問されます(実話).
\end{answer}
\end{problem}

\begin{problem}
満月の夜に近所の神社で何をすると良いですか?
\begin{answer}
おみくじなどを引くと良いでしょう.
\end{answer}
\end{problem}

\section{大晦日には}
\begin{Filesave}{\answersfile}\showanswersubsection\end{Filesave}

\subsection{蕎麦を食べたい}

\subsubsection{一体どうするのか}

どうにもならない.

\subsubsection{どうしても蕎麦を食べたい}

\begin{problem}
結局どうするのですか?
\begin{answer}
蕎麦を買いに行きます.
\end{answer}
\end{problem}

\Closesolutionfile{\answersfile}
\appendix
\Readsolutionfile{\answersfile}

%%%%%%%%%%%%%%%%%%%%%%%%%%%%%%%%%%%%%%%%%%%%%%%%%%%%%%%%%%%%%%%%%%%%%%%%%%%%%%
\end{document}
