% !TEX program = LuaLaTeX
%%%%%%%%%%%%%%%%%%%%%%%%%%%%%%%%%%%%%%%%%%%%%%%%%%%%%%%%%%%%%%%%%%%%%%%%%%%%%%
%\def\nosolutionfiles{} % 定義すると \usepackage[nosolutionfiles]{answers} になる.
\def\TITLE{answers.styの使用例}
\def\AUTHOR{黒木 玄}
\def\DATE{2024年12月31日}
%%%%%%%%%%%%%%%%%%%%%%%%%%%%%%%%%%%%%%%%%%%%%%%%%%%%%%%%%%%%%%%%%%%%%%%%%%%%%%
\documentclass[%
  lualatex,
  ja=standard,
  textwidth-limit=60,
  %a4paper,
  b5paper,
  %12pt,
  oneside,
  everyparhook=compat]{bxjsbook}

%%% amsthm
\usepackage{amsthm}
\theoremstyle{definition}
\newtheorem{problem}{問題}[chapter]

%%% answers
\ifdefined\nosolutionfiles
  \usepackage[nosolutionfiles]{answers}
\else
  \usepackage{answers}
\fi
\newcommand\answersfile{bookanswers}
\Newassociation{answer}{showanswer}{\answersfile}
\renewcommand{\showanswerlabel}[1]{\hspace{4.6pt}\textbf{問題#1解答例.}}

\newcommand\showanswerpart{%
  \part{問題の解答例}
  \newcounter{answerchapter}
  \setcounter{answerchapter}{0}
}

\newcommand\showanswerchapter{%
  \stepcounter{answerchapter}
  \chapter{第\theanswerchapter 章の問題の解答例}
}

%%% maketitle
\title{\Huge{\textbf{\TITLE}}}
\author{\LARGE{\textbf{\AUTHOR}}}
\date{\Large{\textbf{\DATE}}}

%%%%%%%%%%%%%%%%%%%%%%%%%%%%%%%%%%%%%%%%%%%%%%%%%%%%%%%%%%%%%%%%%%%%%%%%%%%%%%
\begin{document}

%%% タイトルと著者など
\maketitle

%%% 序文と目次
\frontmatter

\chapter*{序文}
\addcontentsline{toc}{chapter}{序文}

これはanswers.styの使用例です.

\setcounter{tocdepth}{2}
\tableofcontents
\addcontentsline{toc}{chapter}{目次}

%%% 本文
\mainmatter

\Opensolutionfile{\answersfile}
\begin{Filesave}{\answersfile}\showanswerpart\end{Filesave}

\part{満月に注意せよ}

\chapter{深夜でも明るい}
\begin{Filesave}{\answersfile}\showanswerchapter\end{Filesave}

\section{散歩するとどうなるか}

\begin{problem}
満月の深夜に近所の神社に散歩に行くとどうなりますか?
\begin{answer}
満月の深夜に近所の神社に散歩に行くと, 警官に見付かって職務質問されます(実話).
\end{answer}
\end{problem}

\begin{problem}
満月の夜に近所の神社で何をすると良いですか?
\begin{answer}
おみくじなどを引くと良いでしょう.
\end{answer}
\end{problem}

\part{大晦日には}

\chapter{蕎麦を食べたい}
\begin{Filesave}{\answersfile}\showanswerchapter\end{Filesave}

\section{一体どうするのか}

どうにもならない.

\section{どうしても蕎麦を食べたい}

\begin{problem}
結局どうするのですか?
\begin{answer}
蕎麦を買いに行きます.
\end{answer}
\end{problem}

\Closesolutionfile{\answersfile}
\Readsolutionfile{\answersfile}

%%%%%%%%%%%%%%%%%%%%%%%%%%%%%%%%%%%%%%%%%%%%%%%%%%%%%%%%%%%%%%%%%%%%%%%%%%%%%%
\end{document}
